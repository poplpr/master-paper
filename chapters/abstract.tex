%%
% The BIThesis Template for Graduate Thesis
%
% Copyright 2020-2023 Yang Yating, BITNP
%
% This work may be distributed and/or modified under the
% conditions of the LaTeX Project Public License, either version 1.3
% of this license or (at your option) any later version.
% The latest version of this license is in
%   https://www.latex-project.org/lppl.txt
% and version 1.3 or later is part of all distributions of LaTeX
% version 2005/12/01 or later.
%
% This work has the LPPL maintenance status `maintained'.
%
% The Current Maintainer of this work is Feng Kaiyu.

\begin{abstract}
  近年来,生成性大语言模型(LLMs)迅速发展,生成的内容几乎无法与人类撰写的文本区分开来。尽管这一进展在各个领域得到了广泛应用,但也引发了教育工作者对学生提交作品真实性的重大担忧。因此,解决教育领域中对人工智能生成文本(AIGT)的滥用问题已成为当务之急。目前的检测策略主要集中在整篇文档上,这并未完全满足实际需求。由于学生在将AI生成的内容纳入自己的论文之前,可能会对其进行一定程度的修改,因此细粒度检测,特别是在句子级别的检测,显得尤为重要。因此,追踪文本来源的任务越来越受到关注。针对这一点,本研究创新性地提出了教育领域内文本来源追踪的任务,并构建了相应的数据集,命名为模型改写文本检测数据集,英文名称为 TOSWT(Tracing the Origins of Students’ Writing Texts)。该数据集包含由五个杰出的语言模型生成的文本,基于学生撰写的论证性论文,共包含53,328个文档级和147,976个句子级的数据样本。研究通过在文档级和句子级数据上进行实验评估,评估了多种深度学习检测模型。结果表明,文本来源追踪的任务极具挑战性,其中句子级任务尤其困难。

  为便于用户操作,本文还实现了一个文本改写检测系统。需求分析、系统架构设计以及各模块和接口的设计均在文中进行了详细阐述。该系统的实现基于Flask框架,前端采用Vue.js进行开发,后端则使用Python编程语言。系统的主要功能包括文本改写和文本改写检测。该系统的设计与实现为后续研究奠定了坚实的基础。。
\end{abstract}

% 如需手动控制换行连字符位置,可写 aa\-bb,详见
% https://bithesis.bitnp.net/faq/hyphen.html

\begin{abstractEn}
  In recent years, generative large language models (LLMs) have undergone rapid development, producing content that is nearly indistinguishable from human-written text. While this advancement has found widespread application across various fields, it has also raised significant concerns among educators regarding the authenticity of student submissions. Consequently, addressing the misuse of AI-generated text (AIGT) in the educational sector has become an urgent priority. Current detection strategies primarily focus on whole documents, which do not fully satisfy practical requirements. Due to the likelihood that students may modify AI-generated content to some extent before incorporating it into their essays, fine-grained detection, particularly at the sentence level, is of paramount importance. Consequently, the task of tracing text provenance has increasingly garnered attention. In light of this, this study innovatively proposes the task of text provenance tracing within the educational domain and constructs a corresponding dataset named TOSWT (Tracing the Origins of Students’ Writing Texts). This dataset, which comprises texts generated by five outstanding large language models, is based on argumentative essays written by students and contains a total of 53,328 document-level and 147,976 sentence-level data samples. The study evaluates multiple deep learning detection models through experimental assessments on both document-level and sentence-level data. The results indicate that the task of text provenance tracing is highly challenging, with the sentence-level task proving particularly difficult.

  To facilitate user interaction, this study also implements a text rewriting detection system. The requirements analysis, system architecture design, as well as the design of various modules and interfaces are elaborately described within this document. The implementation of the system is based on the Flask framework, with the front end developed using Vue.js and the back end utilizing the Python programming language. The primary functionalities of the system include text rewriting and text rewriting detection. The design and implementation of this system provide a solid foundation for future research.
\end{abstractEn}
