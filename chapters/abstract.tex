%%
% The BIThesis Template for Graduate Thesis
%
% Copyright 2020-2023 Yang Yating, BITNP
%
% This work may be distributed and/or modified under the
% conditions of the LaTeX Project Public License, either version 1.3
% of this license or (at your option) any later version.
% The latest version of this license is in
%   https://www.latex-project.org/lppl.txt
% and version 1.3 or later is part of all distributions of LaTeX
% version 2005/12/01 or later.
%
% This work has the LPPL maintenance status `maintained'.
%
% The Current Maintainer of this work is Feng Kaiyu.

\begin{abstract}
  生成式语言模型迅速的发展,促使生成的内容几乎无法与人类撰写的文本区分。这引发了教育工作者对学生提交作品真实性的重大担忧。因此,解决教育领域中对人工智能生成文本(AIGT)的滥用问题已成为当务之急。目前的检测策略主要集中在整篇文档上,这并未完全满足实际需求。由于学生在将AI生成的内容纳入自己的论文之前,可能会对其进行一定程度的修改,因此细粒度检测,特别是在句子级别的检测,显得尤为重要。因此,检测生成文本语言模型的任务越来越受到关注。针对以上问题,本研究提出了教育领域内生成文本语言模型检测的任务,取得了如下研究成果:
  
  (1)针对目前模型改写文本检测数据集仅能分辨是否由大语言模型生成而无法判断由哪个特定大语言模型生成,以及细粒度文本数据集不足这两个问题,构建了相应的数据集,命名为模型改写文本检测数据集,英文名称为 TOSWT(Tracing the Origins of Students’ Writing Texts)。该数据集包含由五个杰出的语言模型生成的文本,基于学生撰写的论证性论文,共包含53,328个文档级和147,976个句子级的数据样本。

  (2)研究通过在文档级和句子级数据上进行实验评估,评估了多种基于深度学习的模型改写文本检测模型,并提出一种基于句子级文本检测技术的文档级文本检测方法,该方法在文档级文本数据中达到较好效果,具有 78.26\% 的准确率。最终结果表明,模型改写文本检测的任务极具挑战性,其中句子级任务尤其困难。
  
  (3)为便于用户操作,本文还实现了一个文本改写检测系统。需求分析、系统架构设计以及各模块和接口的设计均在文中进行了详细阐述。该系统的实现基于Flask框架,前端采用Vue.js进行开发,后端则使用Python编程语言。系统的主要功能包括文本改写和文本改写检测。该系统的设计与实现为后续研究奠定了坚实的基础。
\end{abstract}

% 如需手动控制换行连字符位置,可写 aa\-bb,详见
% https://bithesis.bitnp.net/faq/hyphen.html

\begin{abstractEn}
  The rapid advancement of generative language models has made AI-generated content nearly indistinguishable from human-written text. This has raised significant concerns among educators regarding the authenticity of students' submitted work. Consequently, addressing the misuse of AI-generated text (AIGT) in education has become an urgent priority. Current detection strategies primarily focus on entire documents, which do not fully meet practical needs. Since students may modify AI-generated content before incorporating it into their essays, fine-grained detection---particularly at the sentence level---is crucial. As a result, the task of detecting generative text has garnered increasing attention.
  
  To address these challenges, this study proposes a task for detecting AI-generated text in the educational domain and achieves the following key contributions:
  
  (1) Dataset Construction: Existing datasets for rewritten text detection can only determine whether a text is generated by a large language model (LLM) but fail to identify the specific source model. Additionally, there is a lack of fine-grained text datasets. To mitigate these issues, we construct a new dataset named TOSWT (Tracing the Origins of Students' Writing Texts), which includes texts generated by five prominent language models based on students' argumentative essays. The dataset comprises 53,328 document-level and 147,976 sentence-level samples.
  
  (2) The study conducted experimental evaluations on both document-level and sentence-level datasets to assess multiple deep learning-based models for detecting rewritten text, while proposing a novel document-level detection method that incorporates sentence-level detection results. The final findings demonstrate that the task of detecting rewritten text is highly challenging, with sentence-level detection proving particularly difficult.
  
  (3) System Implementation: To facilitate user interaction, we develop a text rewriting detection system. The paper elaborates on requirement analysis, system architecture design, and the implementation of modules and interfaces. The system is built using the Flask framework for the backend (Python) and Vue.js for the frontend. Its core functionalities include text rewriting and text rewriting detection, providing a robust foundation for future research.
\end{abstractEn}
