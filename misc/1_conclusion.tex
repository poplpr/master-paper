%%
% The BIThesis Template for Graduate Thesis
%
% Copyright 2020-2023 Yang Yating, BITNP
%
% This work may be distributed and/or modified under the
% conditions of the LaTeX Project Public License, either version 1.3
% of this license or (at your option) any later version.
% The latest version of this license is in
%   https://www.latex-project.org/lppl.txt
% and version 1.3 or later is part of all distributions of LaTeX
% version 2005/12/01 or later.
%
% This work has the LPPL maintenance status `maintained'.
%
% The Current Maintainer of this work is Feng Kaiyu.

\begin{conclusion}

一、本文工作

本文针对教育领域的人工智能文本滥用、目前模型改写文本检测数据集细粒度数据不足等问题,提出了新的研究框架并创建了模型改写文本检测数据集,并构建了模型改写文本检测系统。微调的 DeBERTa-V3-Large 模型在文档级任务上表现优异,显示出其在处理复杂文本结构方面的潜力。本文主要贡献如下:

(1)提出了一个新的模型改写文本检测数据集,包含了多种大语言模型生成的文本样本,以句子的细粒度涵盖了多种模型改写文本。该数据集为研究人员提供了一个有价值的资源,以评估和比较不同的文本检测方法。

(2)测试了若干模型改写文本检测模型,能够有效地识别和分类不同大语言模型生成的文本。通过对模型的微调和优化,系统在文档级任务上取得了显著的成效。但是在句子级模型改写文本检测任务上仍然存在一定的挑战性,表明该领域仍有进一步研究的空间。

(3)搭建了一个模型改写文本检测系统,提供了用户友好的界面和高效的文本检测功能。该系统可以帮助教育工作者和研究人员快速识别和分析学生的写作风格和文本来源,为教育领域的文本检测提供了实用的解决方案。

二、未来工作与展望

本文针对目前的大语言模型生成文本检测数据集的细粒度数据不足、无法检测出一段文本具体由哪种大语言模型生成的问题,提出了模型生成文本检测数据集,但仍存在进一步探索空间,未来工作可以从以下方面进行改进:

(1)未来研究可重点优化句子级文本检测,并构建更丰富的数据集来提升模型性能。具体可从两方面着手:一是深入分析生成文本的词汇、句法和语义特征以提高检测精度;二是建立覆盖多领域、多语言的数据集来增强泛化能力。同时,随着生成模型的快速进化,检测方法需持续创新,可引入在线学习或元学习策略来动态适应新型文本。此外,还需优化模型计算效率,平衡实时场景下的检测速度与准确性。这些改进将推动文本检测技术的实际应用。

(2)此外,结合其他技术(如图神经网络或迁移学习)可能为这一领域带来新的突破。图神经网络能够有效捕捉文本中复杂的语义关系网络,有助于识别生成文本中隐藏的模式特征;而迁移学习则可以利用预训练语言模型的知识,快速适应不同领域的文本检测任务。通过融合这些先进技术,不仅可以提升模型对生成文本的判别能力,还能增强其在跨领域、跨语言场景下的适应性和鲁棒性。这种多技术融合的研究思路,将为生成文本检测开辟更广阔的发展空间。

(3)本文搭建的系统目前仍存在一些不足之处,如功能过少、管理员仍需要从命令行操作用户等。未来的开发可以集中在改进系统的用户体验和功能扩展上,以满足更广泛的需求。首先,可以开发更直观的图形化管理界面,让管理员能够通过可视化操作完成用户管理、权限设置等日常工作,减少对命令行操作的依赖。其次,系统功能模块需要进一步丰富,比如增加数据分析、自动化报告生成等实用功能,提升系统的整体价值。此外,响应速度和操作流畅度也有待优化,通过改进前后端交互机制和缓存策略,可以显著提升用户的使用体验。这些改进方向将帮助系统更好地服务于目标用户群体。

\end{conclusion}
