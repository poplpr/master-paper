%%
% The BIThesis Template for Graduate Thesis
%
% Copyright 2020-2023 Yang Yating, BITNP
%
% This work may be distributed and/or modified under the
% conditions of the LaTeX Project Public License, either version 1.3
% of this license or (at your option) any later version.
% The latest version of this license is in
%   https://www.latex-project.org/lppl.txt
% and version 1.3 or later is part of all distributions of LaTeX
% version 2005/12/01 or later.
%
% This work has the LPPL maintenance status `maintained'.
%
% The Current Maintainer of this work is Feng Kaiyu.

\begin{conclusion}

本研究针对教育领域的文本溯源任务,提出了新的研究框架并创建了模型改写文本检测数据集。我们的实验结果表明,经过微调的 DeBERTa-V3-Large 模型在文档级任务上表现优异,显示出其在处理复杂文本结构方面的潜力。

然而,在句子级任务中,该模型的表现相对较差,揭示了这一领域仍然存在的挑战。本研究的贡献在于为文本溯源提供了新的数据集和基准,推动了对 AI 生成文本检测技术的理解。尽管如此,研究也存在一定的局限性,例如数据集规模和多样性可能影响模型的泛化能力。

本文搭建的模型改写文本检测系统为教育工作者和研究人员提供了一个有价值的工具,帮助他们识别和分析学生的写作风格和文本来源。通过对模型的进一步优化和改进,我们相信可以在未来实现更高效的文本溯源。

未来的研究可以集中在改进句子级文本溯源的算法,以及探索更大规模和多样化的数据集,以提升模型的整体性能和适应性。此外,结合其他技术(如图神经网络或迁移学习)可能为这一领域带来新的突破。本文搭建的系统目前仍存在一些不足之处,如功能过少、管理员仍需要从命令行操作用户等。未来的开发可以集中在改进系统的用户体验和功能扩展上,以满足更广泛的需求。

\end{conclusion}
